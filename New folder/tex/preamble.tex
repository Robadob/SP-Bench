\usepackage[T1]{fontenc}
\usepackage[latin9]{inputenc}
\usepackage{geometry}
\geometry{verbose,tmargin=4cm,bmargin=4cm,lmargin=3cm,rmargin=3cm}
\setcounter{secnumdepth}{3}
\setcounter{tocdepth}{3}
\usepackage[nottoc,notlot,notlof]{tocbibind}
\usepackage{color}
\definecolor{note_fontcolor}{rgb}{0.800781, 0.800781, 0.800781}
\usepackage{verbatim}
\usepackage{float}
\usepackage{wrapfig}
\usepackage{textcomp}
\usepackage{graphicx}
\usepackage{subscript}
\usepackage{pgfgantt}
\usepackage{tabu}

\makeatletter

\floatstyle{ruled}
\newfloat{algorithm}{tbp}{loa}[chapter]
\providecommand{\algorithmname}{Algorithm}
\floatname{algorithm}{\protect\algorithmname}

%%%%%%%%%%%%%%%%%%%%%%%%%%%%%% User specified LaTeX commands.
\usepackage[british]{babel}
\usepackage{pdflscape}
\usepackage{fullpage}
\usepackage{titleps}
\usepackage{wrapfig}
\usepackage{pdflscape}
\usepackage{pgfgantt}
%\newpagestyle{main}{
%\sethead[][Chapter \thechapter. \chaptertitle][]{}{Chapter \thechapter. \chaptertitle}{}}
%\pagestyle{main}
\select@language{british}
\usepackage{fancyhdr}
\usepackage{listings}
\usepackage{xcolor}
\usepackage{textcomp}
\usepackage{setspace}
\setstretch{1.0}
\lstset{
language=[Visual]C++,
keywordstyle=\bfseries\ttfamily\color[rgb]{0,0,1},
keywordstyle=[2]\bfseries\ttfamily\color[rgb]{0,0,1},
identifierstyle=\ttfamily,
commentstyle=\color[rgb]{0.133,0.545,0.133},
stringstyle=\ttfamily\color[rgb]{0.627,0.126,0.941},
showstringspaces=false,
basicstyle=\small,
tabsize=2,
breaklines=true,
prebreak = \raisebox{0ex}[0ex][0ex]{\ensuremath{\hookleftarrow}},
breakatwhitespace=false,
aboveskip={1.5\baselineskip},
columns=fixed,
upquote=true,
extendedchars=true,
morekeywords={test}
keywords=[2]{__restrict__,__ldg},
}
\lstdefinelanguage{GLSL}
{
sensitive=true,
morekeywords=[1]{
attribute, const, uniform, varying,
layout, centroid, flat, smooth,
noperspective, break, continue, do,
for, while, switch, case, default, if,
else, in, out, inout, float, int, void,
bool, true, false, invariant, discard,
return, mat2, mat3, mat4, mat2x2, mat2x3,
mat2x4, mat3x2, mat3x3, mat3x4, mat4x2,
mat4x3, mat4x4, vec2, vec3, vec4, ivec2,
ivec3, ivec4, bvec2, bvec3, bvec4, uint,
uvec2, uvec3, uvec4, lowp, mediump, highp,
precision, sampler1D, sampler2D, sampler3D,
samplerCube, sampler1DShadow,
sampler2DShadow, samplerCubeShadow,
sampler1DArray, sampler2DArray,
sampler1DArrayShadow, sampler2DArrayShadow,
isampler1D, isampler2D, isampler3D,
isamplerCube, isampler1DArray,
isampler2DArray, usampler1D, usampler2D,
usampler3D, usamplerCube, usampler1DArray,
usampler2DArray, sampler2DRect,
sampler2DRectShadow, isampler2DRect,
usampler2DRect, samplerBuffer,
isamplerBuffer, usamplerBuffer, sampler2DMS,
isampler2DMS, usampler2DMS,
sampler2DMSArray, isampler2DMSArray,
usampler2DMSArray, struct},
morekeywords=[2]{
radians,degrees,sin,cos,tan,asin,acos,atan,
atan,sinh,cosh,tanh,asinh,acosh,atanh,pow,
exp,log,exp2,log2,sqrt,inversesqrt,abs,sign,
floor,trunc,round,roundEven,ceil,fract,mod,modf,
min,max,clamp,mix,step,smoothstep,isnan,isinf,
floatBitsToInt,floatBitsToUint,intBitsToFloat,
uintBitsToFloat,length,distance,dot,cross,
normalize,faceforward,reflect,refract,
matrixCompMult,outerProduct,transpose,
determinant,inverse,lessThan,lessThanEqual,
greaterThan,greaterThanEqual,equal,notEqual,
any,all,not,textureSize,texture,textureProj,
textureLod,textureOffset,texelFetch,
texelFetchOffset,textureProjOffset,
textureLodOffset,textureProjLod,
textureProjLodOffset,textureGrad,
textureGradOffset,textureProjGrad,
textureProjGradOffset,texture1D,texture1DProj,
texture1DProjLod,texture2D,texture2DProj,
texture2DLod,texture2DProjLod,texture3D,
texture3DProj,texture3DLod,texture3DProjLod,
textureCube,textureCubeLod,shadow1D,shadow2D,
shadow1DProj,shadow2DProj,shadow1DLod,
shadow2DLod,shadow1DProjLod,shadow2DProjLod,
dFdx,dFdy,fwidth,noise1,noise2,noise3,noise4,
EmitVertex,EndPrimitive},
morekeywords=[3]{
gl_VertexID,gl_InstanceID,gl_Position,
gl_PointSize,gl_ClipDistance,gl_PerVertex,
gl_Layer,gl_ClipVertex,gl_FragCoord,
gl_FrontFacing,gl_ClipDistance,gl_FragColor,
gl_FragData,gl_MaxDrawBuffers,gl_FragDepth,
gl_PointCoord,gl_PrimitiveID,
gl_MaxVertexAttribs,gl_MaxVertexUniformComponents,
gl_MaxVaryingFloats,gl_MaxVaryingComponents,
gl_MaxVertexOutputComponents,
gl_MaxGeometryInputComponents,
gl_MaxGeometryOutputComponents,
gl_MaxFragmentInputComponents,
gl_MaxVertexTextureImageUnits,
gl_MaxCombinedTextureImageUnits,
gl_MaxTextureImageUnits,
gl_MaxFragmentUniformComponents,
gl_MaxDrawBuffers,gl_MaxClipDistances,
gl_MaxGeometryTextureImageUnits,
gl_MaxGeometryOutputVertices,
gl_MaxGeometryOutputVertices,
gl_MaxGeometryTotalOutputComponents,
gl_MaxGeometryUniformComponents,
gl_MaxGeometryVaryingComponents,gl_DepthRange},
morecomment=[l]{//},
morecomment=[s]{/*}{*/},
morecomment=[l][keywordstyle4]{\#},
}
\lstset{
backgroundcolor=\color[rgb]{0.95, 0.95, 0.95},
tabsize=2,
rulecolor=,
basicstyle=\scriptsize,
upquote=true,
aboveskip={1.5\baselineskip},
columns=fixed,
showstringspaces=false,
extendedchars=true,
breaklines=true,
prebreak = \raisebox{0ex}[0ex][0ex]{\ensuremath{\hookleftarrow}},
frame=single,
showtabs=false,
showspaces=false,
showstringspaces=false,
identifierstyle=\ttfamily,
keywordstyle=\color[rgb]{1.0,0,0},
keywordstyle=[1]\color[rgb]{0,0,0.75},
keywordstyle=[2]\color[rgb]{0.5,0.0,0.0},
keywordstyle=[3]\color[rgb]{0.127,0.427,0.514},
keywordstyle=[4]\color[rgb]{0.4,0.4,0.4},
commentstyle=\color[rgb]{0.133,0.545,0.133},
stringstyle=\color[rgb]{0.639,0.082,0.082},
}
\usepackage{makecell}
\renewcommand\cellgape{\Gape[2pt]}

\@ifundefined{showcaptionsetup}{}{%
 \PassOptionsToPackage{caption=true}{subfig}}
\usepackage{subfig}
\makeatother

\usepackage{babel}
\usepackage{listings}
\renewcommand{\lstlistingname}{Listing}

\usepackage{titling}
\newcommand{\subtitle}[1]{%
  \posttitle{%
    \par\end{center}
    \begin{center}\Large#1\end{center}
    \vskip0.5em}%
}
%Pete's Todo stuff
\usepackage{xifthen}
\usepackage{mdframed}
\definecolor{todo_light}{HTML}{FFCC7D}
\definecolor{todo_dark}{HTML}{CB8F00}
\newmdenv[
    linecolor=todo_dark,
    linewidth=12pt,
    backgroundcolor=todo_light,
    frametitlebackgroundcolor=todo_light,
    innertopmargin=6pt,
    innerrightmargin=4pt,
    innerbottommargin=6pt,
    innerleftmargin=4pt,
    skipabove=12pt,
    skipbelow=12pt,
    topline=false,
    rightline=false,
    bottomline=false
]{md_todo}
\newcommand{\todo}[2][]{
    \begin{md_todo}[]
    \texttt{@todo:} \textbf{#1} \ifthenelse{\equal{#1}{}}{}{\newline\noindent}#2
    \end{md_todo}
}
%Pete's Todo stuff, but adjusted for a mint coloured note
\usepackage{xifthen}
\usepackage{mdframed}
\definecolor{note_light}{HTML}{82CCB5}
\definecolor{note_dark}{HTML}{089D7E}
\newmdenv[
    linecolor=note_dark,
    linewidth=12pt,
    backgroundcolor=note_light,
    frametitlebackgroundcolor=note_light,
    innertopmargin=6pt,
    innerrightmargin=4pt,
    innerbottommargin=6pt,
    innerleftmargin=4pt,
    skipabove=12pt,
    skipbelow=12pt,
    topline=false,
    rightline=false,
    bottomline=false
]{md_note}
\newcommand{\note}[2][]{
    \begin{md_note}[]
    \textbf{#1} \ifthenelse{\equal{#1}{}}{}{\newline\noindent}#2
    \end{md_note}
}
%Make paragraph command an italics subsection
\makeatletter
\renewcommand{\paragraph}{%
  \@startsection{paragraph}{4}{0mm}%
      {-.5\baselineskip}%
      {.1\baselineskip}%
  {\normalfont\normalsize\itshape}%
}
\makeatother
%define unchapter for headerless chapter
\makeatletter
\newcommand{\unchapter}[1]{%
  \begingroup
  \let\@makechapterhead\@gobble % make \@makechapterhead do nothing
  \chapter{#1}
  \endgroup
}
\makeatother
\usepackage{url}
\usepackage{hyperref}
\hypersetup{
    colorlinks,
    citecolor=black,
    filecolor=black,
    linkcolor=black,
    urlcolor=black
}
%Glossaries
\usepackage[automake,acronym,section=chapter,hyperfirst=true]{glossaries}
\makeglossaries

\usepackage{pdfpages}
\usepackage{fmtcount}

\setlength{\intextsep}{0pt}%reduces gap under wrapped figures