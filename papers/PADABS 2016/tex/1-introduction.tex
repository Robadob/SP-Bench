
\section{Introduction}
  Many \glspl{abs} contain mobile agents located within a continuous space. These agents may represent entities such as particles, people and vehicles. \textbf{Provide some actual examples of related ABSs? e.g. sph, pedestrian).} In order for these mobile agents to progress between states within a simulation, they must be aware of their neighbouring agents. This awareness is provided by a fixed radius nearest neighbour search, whereby each agent considers the properties of every other agent located within a radius about their location. This searched area can be considered the agent's neighbourhood.
  
  This neighbourhood search must be carried out every timestep of a simulation, ensuring an agent has access to the most recent information about their neighbourhood. Consequently this causes the neighbourhood search to often occupy the majority of execution time within affected models, due to the large amount of scattered memory accesses that are performed \textbf{ (don't really like how long this sentence is, but can't decide how split it)}. Therefore it is important to ensure that neighbourhood searches within \glspl{abs} and frameworks are performant.
  
  This paper introduces a benchmark model named `circles', which has been designed to assess the performance of neighbourhood searches, providing an overview of the benchmark's applications alongside a preliminary comparison of results obtained from: D-MASON, FLAMEGPU, REPAST and a bespoke uniform spatial partitioning implementation.\textbf{This paragraph needs work. }\textbf{Could elaborate further on the applications of the benchmark.}
  
  The results within this paper assess each frameworks implementation against each of the metrics which can be levied from the circles benchmark. These results show...\textbf{Something clearer with regard to the results provided within the paper.}
  
  The \gls{openab} is an initiative to pool the research community's \gls{abs} knowledge and resources. The aim is for this collaboration to provide a centralised repository of agent benchmark models and results collected across a wide range of simulation frameworks and hardware architectures. The model and results presented within this paper are also available through the \gls{openab} website\footnote{\url{http://www.openab.org}}, whereby contribution details can also be found for those wishing to submit their own results.
  
  The remainder of this paper is organised as follows: Section \ref{sec:related-work} provides an overview of related research; Section \ref{sec:benchmark-model} lays out a clear specification of the `circles' benchmark model and how it can be utilised effectively; Section \ref{sec:assessed-frameworks} details the frameworks which have been assessed using the benchmark; Section \ref{sec:results} discusses the results obtained from the application of the `circles' benchmark to each framework; Finally Section \ref{sec:conclusion} presents the concluding remarks and directions for further research.
