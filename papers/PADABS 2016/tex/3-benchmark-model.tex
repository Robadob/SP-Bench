\section{Benchmark Model\label{sec:benchmark-model}}
\textit{Don't mention the origin for the benchmark from FLAMEGPU, as this somewhat outs the anonymity}
  The `circles' benchmark model is designed to utilise neighbourhood search in a manner analogous to a simplified particle simulation in two or three dimensions (although it could easily be extended to higher levels of dimensionality if required). Within the model each agent represents a particle whose motion is driven by forces applied from particles within their local neighbourhood.
    
  \subsection{Specification}
    The benchmark model is configured using the parameters in Table \ref{tab:benchmark-parameters}. In addition to these parameters 

    \begin{table}
      \begin{tabu}{ |c|>{\raggedright}X| }
        \hline
        \textbf{Parameter} & \textbf{Description} \\ \hline
        $k_{rep}$ & The repulsive dampening argument. Increasing this value encourages agents to repel. \\ \hline
        $k_{att}$ & The attractive dampening argument. Increasing this value encourages agents to attract. \\ \hline
        $r$ & The interaction radius. Increasing this value increases the radius of the neighbourhoods searched, subsequently increasing agent communication. \\ \hline
        $ \rho $ & The density of agents within the environment. \\ \hline
        $W$ & The diameter of the environment, this value is shared by each dimension therefore in a two dimensional environment it represents the width and height. Increasing this value is equivalent to increasing the scale of the problem (e.g. the number of agents) assuming $ \rho $ remains unchanged.\\ \hline
      \end{tabu}
      \caption{\label{tab:benchmark-parameters}The parameters for configuring the `circles' benchmark model.}
    \end{table}
  
  \subsection{Effective Usage}
    %Details of what the benchmark assesses and how changing parameters leads to assessment of different metrics.
   
    %Lightly touch on metrics the benchmark is incapable of assessing.
  \subsection{Limitations}
    %Discussion of tertiary benchmarks or extensions to �circles� that may be used in future to assess the metrics current benchmark is incapable of assessing.