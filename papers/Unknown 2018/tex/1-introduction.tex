\section{Introduction}  
  %These two paragraphs from PADABS fit, but can't self plagorise
  %\IEEEPARstart{M}{any} complex systems have mobile entities located within a continuous space such as: particles, people or vehicles. Typically these systems are represented via \gls{abs} where entities are agents. In order for these mobile agents to decide actions, they must be aware of their neighbouring agents. This awareness is typically provided by fixed radius near neighbours (FRNNs) search, whereby each agent considers the properties of every other agent located within a spatial radial area about their simulated position. This searched area can be considered the agent's neighbourhood and must be searched every timestep of a simulation, ensuring the agent has access to the most recent information about their neighbourhood. In many cases such as flocking, pedestrian interaction and cellular systems, the majority of time is spent performing this neighbourhood search, as opposed to agent logic. It is hence often the primary performance limitation.
  
  %The most common technique utilised for accelerating FRNNs is one of uniform spatial partitioning. Within uniform spatial partitioning, the environment is decomposed into a regular grid, partitioned according to the interaction radius. Agents are then stored or sorted according to the grid cell they are located within. Agents consider their neighbourhood by performing a distance test on all agents within their own grid partition and any directly adjacent neighbouring grid cells. This has caused researchers to seek to improve the efficiency of FRNNs handling, primarily by approaching more efficient memory access patterns \cite{GS*10,Hoe14,HY*15}. However without a rigorous standard to compare implementations, exposing their relative benefits is greatly complicated.
  
  \glsunset{frnns} 
  \IEEEPARstart{F}{ixed} radius near neighbours (FRNNs) is the process whereby a search is performed across messages containing spatial locations, to survey all those messages which occur within a fixed radius of the search location. This process is used in the representation of complex systems to provide spatial awareness to the mobile entities such as: particles, people or vehicles. These entities exist in a continuous space and require awareness of neighbouring entities or events.
  
  Due to the nature of complex systems, containing thousands of similar interacting entities, their computation is well suited for utilising \glspl{gpu}. The highly parallel \gls{simt} architecture present in \glspl{gpu} permits millions of entities to be processed simultaneously in fractions of a second. The technique of \gls{usp} is most commonly used to provide efficient \gls{frnns} on \gls{gpu} hardware. The \gls{usp} data structure is constructed using the thoroughly optimised \gls{gpu} primitive operations sort and scan. Similarly accessing messages stored within the data structure is possible with minimal branch divergence which, well optimised for \gls{gpu} allows the \gls{usp} data structure to be efficiently constructed.
  
  Despite \gls{usp} being the most suitable technique, it is still one of the most expensive operations, often requiring more than any internal model logic. The act of every thread accessing a unique subset of the messages simultaneously creates an inevitable memory bottleneck. This has lead researchers to seek out techniques to improve the performance of \gls{frnns}\cite{GS*10,Hoe14,HY*15}, however \gls{usp} despite it's expense.
  %Where did I see USP used for K-NN?
  
  This paper presents novel optimisations to the manner in which bin access is handled under \gls{usp} on \glspl{gpu}. The optimisations presented are implementation agnostic, capable of general application to uniform spatial partitioning as presented within. Specification of the optimisations with suggestions for their implementation are provided alongside a comparison of results obtained between implementations before and after optimisation. 
  
  The results within this paper assess the performance according to the metrics of problem size and neighbourhood size in both a static diagnostic model and a benchmark model representative of a physical simulation.
  \note{Should probably tease results in the above paragraph}
  
  The remainder of this paper is organised as follows: Section \ref{sec:spatial-partitioning} provides an overview of available techniques for performing \glsentrylong{frnns}, the technique of \glsentrylong{usp} and prior techniques for it's optimisation; Section \ref{sec:innovation} lays out a clear specification of the optimisations to \glsentrylong{usp} and the logic behind their purpose; Section \ref{sec:results} discusses the results obtained when comparing performance before and after the optimisation has been applied to both a diagnostic model and a model representative of a physical simulation; Finally Section \ref{sec:conclusion} presents the concluding remarks and directions for further research.
  
  %FRNNs is the process whereby  
  %This is used in the representation of complex systems
  %The representation of complex system is well suited to the GPU architecture
  %USP is used to provide FRNNs efficiently on the GPU
  
  %Despite USP being efficient, it is still one of the most expensive operations, often requiring more than any internal model logic.
  %So researchers have tried to optimise it