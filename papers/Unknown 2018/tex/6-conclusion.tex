\section{Conclusion\label{sec:conclusion}}
\note{Need to change this into paragraphs}
    \begin{itemize}
      \item The work within this paper has presented two compatible (better word to show they can be combined) optimisations for the accessing of the spatial partitioning data-structure.
      \item Our results show that both individually and combined they out perform an implementation lacking the optimisation in a wide range of problem and neighbourhood sizes (with uniformly/uniform-randomly distributed agents) in two and three dimensions.
      \item The combined approach consistently outperforms or equals performance of the individual optimisations.
      \item The peak improvement in 2D executed in 26.7\% of the runtime of the initial implementation. (300k agents, ~48.0 neighbours)
      \item The peak improvement in 3D executed in 18.6\% of the runtime of the initial implementation. (40k agents, ~46.7 neighbours)
      \item Physical model results showed... This contrast is due to the greater diversity of agent distribution seen(?)...which is an area that will be explored in further detail in the ongoing research....
      \item Further research is ongoing investigating the impact of other variables which may affect the performance of \gls{gpu} uniform spatial partitioning and how they affect these optimisations and may lead to further improvements of performance.
    \end{itemize}